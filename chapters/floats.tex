\chapter{Floating Elements}

\section{Three-Line Tables}

The three-line table format is the recommended style in the *Writing Manual*,
as shown in Table~\ref{tab:exampletable}.
\begin{table}[htbp]
  \centering
  \caption{Caption of the table goes here}
  \label{tab:exampletable}
  \begin{tabular}{cl}
    \toprule
      Operating System & TeX Distribution \\
    \midrule
      All Platforms & TeX Live \\
      macOS & MacTeX \\
      Windows & MikTeX \\
    \bottomrule
  \end{tabular}
  \note{This is a very long table note that demonstrates how lengthy footnotes
  can be displayed properly within a table.}
\end{table}



\section{Long Tables}

Tables that span more than one page should use the dedicated 
\texttt{longtable} environment (see Table~\ref{tab:longtable}).
\begin{longtable}{ccc}
  % First page header
  \caption[Demonstration of Long Table]{Demonstration of Long Table}
  \label{tab:longtable}\\
  \toprule[1.5pt]
    Name  & Description & Remarks\\
  \midrule[1pt]
  \endfirsthead
  % Continued page header
  \caption[]{Demonstration of Long Table (continued)} \\
  \toprule[1.5pt]
    Name  & Description & Remarks \\
  \midrule[1pt]
  \endhead
  % First page footer
  \hline
  \multicolumn{3}{r}{\small Continued on next page}
  \endfoot
  % Last page footer
  \bottomrule[1.5pt]
  \endlastfoot

  AAAAAAAAAAAA & BBBBBBBBBBB & CCCCCCCCCCCCCC \\
  AAAAAAAAAAAA & BBBBBBBBBBB & CCCCCCCCCCCCCC \\ 
  AAAAAAAAAAAA & BBBBBBBBBBB & CCCCCCCCCCCCCC \\ 
  (Repeated rows omitted for brevity)
\end{longtable}



\section{Figures}

Some authors prefer to reference figures and tables by relative position 
(e.g., “the figure below” or “the table above”) and insist that floats 
be placed exactly where they are mentioned.  
In fact, this is not recommended because it can easily create large areas 
of blank space in the text.  
In academic writing, the standard way is to refer to elements 
by their numbers, such as “Figure~\ref{fig:logo}” and 
“Table~\ref{tab:exampletable}”.

\begin{figure}[htbp]
\centering
\includegraphics[width=.3\textwidth]{ouc_logo_fig}
\caption{Sample Figure}
\label{fig:logo}
\end{figure}

For more examples of figures and graphics, many papers on 
\href{https://arxiv.org}{arXiv} provide their \TeX{} source files, 
which can be studied for reference.



\section{Algorithm Environment}

The \texttt{algorithm2e} package is used in this template to typeset algorithms.  
For detailed usage, please refer to the official documentation of the package.

\begin{algorithm}[htbp]
\small
\SetAlgoLined
\KwData{this text}
\KwResult{how to write algorithm with \LaTeX2e }

initialization\;
\While{not at end of this document}{
    read current\;
    \eIf{understand}{
        go to next section\;
        current section becomes this one\;
    }{
        go back to the beginning of current section\;
    }
}
\caption{Example Algorithm 1}
\label{algo:algorithm1}
\end{algorithm}

Note that algorithms can be included in the thesis, 
but inserting large blocks of raw code is generally unwise.  
However, if code must be shown, the \textsf{listings} package 
is recommended for that purpose.
